\documentclass{minimal}
\usepackage{amsmath}
\usepackage{pdflscape}
\usepackage{afterpage}

\begin{document}
\afterpage{\begin{landscape}
Normal-order evaluation:

\begin{eqnarray*}
&& \text{(gcd 206 40)}
\\ &=& \text{(gcd 40 (remainder 206 40))}
\\ &=& \text{(if (= \textbf{(remainder 206 40)} 0) 40 (gcd (remainder 206 40) (remainder 40 (remainder 206 40))))}
\\ &=& \text{(if (= 6 0) a (gcd (remainder 206 40) (remainder 40 (remainder 206 40))))}
\\ &=& \text{(gcd (remainder 206 40) \textbf{(remainder 40 (remainder 206 40))})}
\\ &=& \text{(gcd (remainder 40 (remainder 206 40))}
\\ && \indent \textbf{(remainder (remainder 206 40) (remainder 40 (remainder 206 40))))}
\\ &=& \text{(gcd (remainder (remainder 206 40) (remainder 40 (remainder 206 40)))}
\\ && \indent \textbf{(remainder (remainder 40 (remainder 206 40))}
\\ && \indent\indent\indent\indent \textbf{(remainder (remainder 206 40) (remainder 40 (remainder 206 40)))))}
\\ &=& \textbf{(remainder (remainder 206 40) (remainder 40 (remainder 206 40)))}
\\ &=& \text{(remainder 6 (remainder 40 6))}
\\ &=& \text{(remainder 6 4)}
\\ &=& \text{2}
\end{eqnarray*}

Applicative-order evaluation:

\begin{eqnarray*}
&& \text{(gcd 206 40)}
\\ &=& \text{(gcd 40 \textbf{(remainder 206 40)})}
\\ &=& \text{(gcd 40 6)}
\\ &=& \text{(gcd 6 \textbf{(remainder 40 6)})}
\\ &=& \text{(gcd 6 4)}
\\ &=& \text{(gcd 4 \textbf{(remainder 6 4)})}
\\ &=& \text{(gcd 4 2)}
\\ &=& \text{(gcd 2 \textbf{(remainder 4 2)})}
\\ &=& \text{(gcd 2 0)}
\\ &=& \text{2}
\end{eqnarray*}

When evaluating (gcd 206 40), normal-order evaluation performs 17 remainder
operations, while applicative-order evaluation performs 4 remainder operations.

This is because normal-order evaluation evaluates expressions when their values
are needed and gcd needs to check the value of its second argument in the if
statement's condition. This argument is usually a remainder operation (from the
recursive calls to gcd), so this operation has to be performed before the
normal order evaluation can know what to replace the call to gcd with (either
$a$ or another call to gcd). However, this evaluation is only used for the
outcome of the if staement: the resulting calls to gcd accrue more and more
remainder operations in their first argument, with none of them being evaluated
until their second argument is 0.

With applicative-order evaluation, the remainder operations are performed
before the next call to gcd, instead of happening because the call to gcd
requires their value.

\end{landscape}}
\end{document}
